

% Table of contents formatting
\renewcommand{\contentsname}{\sffamily\Large\bfseries Índice}
\renewcommand{\listfigurename}{\sffamily\Large\bfseries Lista de Figuras}
\renewcommand{\listtablename}{\sffamily\Large\bfseries Lista de Tabelas}
\setcounter{tocdepth}{3}
\usepackage{indentfirst}
\usepackage{listings}
\usepackage{xcolor}
\usepackage{secdot}%Secções com pontos
\usepackage{graphicx}


% Set margins
\usepackage[top=3cm, bottom=2cm, inner=3cm, outer=2.5cm]{geometry}
% \setlength\parindent{0.4in} % indent at start of paragraphs (set to 0.3?)

% Add space between pararaphs
% http://texblog.org/2012/11/07/correctly-typesetting-paragraphs-in-latex/
% \usepackage{parskip}
% \setlength{\parskip}{\baselineskip}

% Headers and page numbering
\usepackage{fancyhdr}
\pagestyle{fancy}

% Following package is used to add background image to front page
\usepackage{wallpaper}

% Table package
\usepackage{ctable}% http://ctan.org/pkg/ctable

% Deal with 'LaTeX Error: Too many unprocessed floats.'
\usepackage{morefloats}
% or use \extrafloats{100}
% add some \clearpage

% % Chapter header
 \usepackage{titlesec, blindtext, color}
 \definecolor{gray89}{rgb}{89,89,89}
% \newcommand{\hsp}{\hspace{15.2pt}}
% \titleformat{\chapter}{\Huge\bfseries\sffamily}{\thechapter\textcolor{gray89}{|}}{0pt}{\Huge\bfseries}

% % Fonts and typesetting
 \setmainfont{Cambria}
 \setsansfont{Trebuchet MS}

% FONTS
\usepackage{xunicode}
\usepackage{xltxtra}
\defaultfontfeatures{Mapping=tex-text} % converts LaTeX specials (``quotes'' --- dashes etc.) to unicode
% \setromanfont[Scale=1.01,Ligatures={Common},Numbers={OldStyle}]{Palatino}
% \setromanfont[Scale=1.01,Ligatures={Common},Numbers={OldStyle}]{Adobe Caslon Pro}
%Following line controls size of code chunks
% \setmonofont[Scale=0.9]{Monaco}
%Following line controls size of figure legends
% \setsansfont[Scale=1.2]{Optima Regular}

%Attempt to set math size
%First size must match the text size in the document or command will not work
%\DeclareMathSizes{display size}{text size}{script size}{scriptscript size}.
\DeclareMathSizes{12}{13}{7}{7}

% ---- CUSTOM AMPERSAND
% \newcommand{\amper}{{\fontspec[Scale=.95]{Adobe Caslon Pro}\selectfont\itshape\&}}

% HEADINGS

\renewcommand{\partname}{Parte}
\titleformat{\part}[display] {\Huge\sffamily\bfseries}{\thispagestyle{fancy}\partname~\thepart}{1em}{}
\titleclass{\chapter}{straight} % no page break
\titleformat{\chapter}{\sffamily\LARGE\bfseries}{\thechapter}{1em}{}
%\titleformat*{\chapter}{\fontsize{16}{1em}\sffamily\bfseries}{\thechapter}{1em}{}
\titleformat*{\section}{\fontsize{14}{1em}\sffamily\bfseries}
\titleformat*{\subsection}{\fontsize{12}{1em}\sffamily\bfseries}

\titlespacing*{\chapter} {0pt}{0pt}{2.3ex plus .2ex}
\titlespacing*{\section} {0pt}{3.25ex plus 1ex minus .2ex}{6pt plus .2ex}
\titlespacing*{\subsection} {0pt}{3.25ex plus 1ex minus .2ex}{6pt plus .2ex}
\setlength{\abovecaptionskip}{6pt plus 0pt minus 2pt}
\setlength{\belowcaptionskip}{12pt plus 0pt minus 2pt}
% \sectionfont{\rmfamily\mdseries\Large}
% \subsectionfont{\rmfamily\mdseries\scshape\normalsize}
% \subsubsectionfont{\rmfamily\bfseries\upshape\normalsize}

\setlength{\cftaftertoctitleskip}{6pt}
\setlength{\cftbeforetoctitleskip}{0pt}
\setlength{\cftafterloftitleskip}{6pt}
\setlength{\cftbeforeloftitleskip}{0pt}
\setlength{\cftafterlottitleskip}{6pt}
\setlength{\cftbeforelottitleskip}{0pt}

\renewcommand{\cftchapfont}{\bfseries \sffamily}
%\renewcommand{\cftsecfont}{\sffamily}
%\renewcommand{\cftsubsecfont}{\sffamily}

% Set figure legends and captions to be smaller sized sans serif font
\usepackage[font={footnotesize,sf}]{caption}

\usepackage{siunitx}

% Adjust spacing between lines to 1.15
\setlength{\parindent}{6mm}%Indentação
\linespread{1.15}%espaçamento
\setlength{\parskip}{6pt}%fim do parágrafo

% Set colour of links to black so that they don't show up when printed
\usepackage{hyperref}
\hypersetup{colorlinks=false, linkcolor=black}

% Tables
\usepackage{booktabs}
\usepackage{threeparttable}
\usepackage{array}
\newcolumntype{x}[1]{%
>{\centering\arraybackslash}m{#1}}%

% Allow for long captions and float captions on opposite page of figures
% \usepackage[rightFloats, CaptionBefore]{fltpage}

% Don't let floats cross subsections
% \usepackage[section,subsection]{extraplaceins}
\setlength{\headheight}{15.2pt}%Altura do cabeçalho - reduzir apenas até 15pt

\renewcommand{\headrulewidth}{0.4pt}%definir linha do cabeçalho
\renewcommand{\footrulewidth}{0pt}%retirar linha do rodapé
\makeatletter
\def\headrule{{\color{gray89}\if@fancyplain\let\headrulewidth\plainheadrulewidth\fi
\hrule\@height\headrulewidth\@width\headwidth
\vskip-\headrulewidth}}
\makeatother

\fancyhead[LE]{\color{gray89} \fontsize{7}{1em} \textsf{Nuno João Casteleira}}%Cabeçalho par
\fancyhead[LO]{}
\fancyhead[RE]{}
\fancyhead[RO]{\color{gray89} \fontsize{7}{1em} \textsf{Tese}}%cabeçalho ímpar
\fancyfoot[C]{\fontsize{11}{1em}\sffamily \thepage}%rodapé

\usepackage{etoolbox}
\patchcmd{\chapter}{\thispagestyle{plain}}{\thispagestyle{fancy}}{}{}

\lstdefinelanguage{swift}
{
  morekeywords={
    func,if,then,else,for,in,while,do,switch,case,default,where,break,continue,fallthrough,return,
    typealias,struct,class,enum,protocol,var,func,let,get,set,willSet,didSet,inout,init,deinit,extension,
    subscript,prefix,operator,infix,postfix,precedence,associativity,left,right,none,convenience,dynamic,
    final,lazy,mutating,nonmutating,optional,override,required,static,unowned,safe,weak,internal,
    private,public,is,as,self,unsafe,dynamicType,true,false,nil,Type,Protocol,
  },
  morecomment=[l]{//}, % l is for line comment
  morecomment=[s]{/*}{*/}, % s is for start and end delimiter
  morestring=[b]" % defines that strings are enclosed in double quotes
}

\definecolor{keyword}{HTML}{BA2CA3}
\definecolor{string}{HTML}{D12F1B}
\definecolor{comment}{HTML}{008400}

\lstset{
  language=swift,
  basicstyle=\small\ttfamily,
  showstringspaces=false, % lets spaces in strings appear as real spaces
  columns=fixed,
  keepspaces=true,
  keywordstyle=\color{keyword},
  stringstyle=\color{string},
  commentstyle=\color{comment},
}

\captionsetup[figure]{labelfont=footnotesize,labelfont=bf,font=footnotesize,textfont={normalfont}}
\renewcommand{\thefigure}{\arabic{figure}}
\renewcommand{\figurename}{Figura}

\captionsetup[table]{labelfont=footnotesize,labelfont=bf,font=footnotesize,textfont={normalfont}}
\renewcommand{\thetable}{\arabic{table}}
\renewcommand{\tablename}{Tabela}

\let\oldthebibliography\thebibliography
\let\endoldthebibliography\endthebibliography
\renewenvironment{thebibliography}[1]{
  \begin{oldthebibliography}{#1}
    \setlength{\itemsep}{8pt}
    \setlength{\parskip}{6pt}
}
{
  \end{oldthebibliography}
}
